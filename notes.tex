\documentclass[letter,12pt]{article}

\usepackage{graphicx}
\usepackage{amsmath, amssymb}
\usepackage[hidelinks]{hyperref}
\usepackage[titles]{tocloft}
\usepackage{placeins}
\usepackage{courier}
\usepackage{multirow, multicol}
\usepackage{soul}

\renewcommand*{\cftdot}{}
\cftpagenumbersoff{subsection}
\cftpagenumbersoff{subsubsection}

\newcommand{\sect}[1]{
\pagebreak
\FloatBarrier
\section*{#1}
\addcontentsline{toc}{section}{\protect\numberline{}#1}%
}
\newcommand{\subsect}[1]{
\FloatBarrier
\subsection*{#1}
\addcontentsline{toc}{subsection}{\protect\numberline{}#1}%
}
\newcommand{\subsubsect}[1]{
\FloatBarrier
\subsubsection*{#1}
\addcontentsline{toc}{subsubsection}{\protect\numberline{}#1}%
}
\newcommand{\eq}[1]{
\begin{equation}
\begin{aligned}
#1
\end{aligned}
\end{equation}
}
\newcommand{\img}[1]{
\begin{figure}[h!]
\begin{center}
\includegraphics[width=0.5\linewidth]{#1}
\end{center}
\end{figure}
}
\newcommand{\bullets}[1]{
\begin{itemize}
#1
\end{itemize}
}
\newcommand{\enum}[1]{
\begin{enumerate}
#1
\end{enumerate}
}
\newcommand{\cols}[2]{
\begin{multicols}{2}
#1

\columnbreak

#2
\end{multicols}
}
\newcommand{\tabl}[2]{
\begin{center}
\begin{tabular}{ | #1 | }
\hline
#2
\hline
\end{tabular}
\end{center}
}

\begin{document}

\title{Advanced Operating Systems Notes}
\author{Madison Hanberry}
\maketitle

\pagebreak

\tableofcontents

\sect{Lesson 2: Introduction}

\subsect{(12) Active Power}

\eq{
P = \frac{1}{2} \times v^2 \times f \times \alpha
}

$C$: Capacitance ($\sim$ Chip area)

$v^2$: Power supply voltage

$f$: Frequency

$\alpha$: Activity factor (\% of transistors active in any given clock cycle)

\subsect{(13) Static Power}

Decreasing \emph{dynamic} power ($v \downarrow$) increases \emph{static} power (increases power leakage).

\img{figures/lesson2_13.png}

\subsect{(14) Quiz: Active Power}

\eq{
P &= 30 W\ at\ 1.0 v, 2\ GHz \\
30 &= \frac{1}{2} C \times (1.0)^2 \times 2 \alpha \\
30 &= C \alpha
}

\subsubsect{(i) $0.9 v, 1.8\ GHz$}

\eq{
P &= \frac{1}{2} (30) (0.9)^2 \times 1.8 \\
P &= 21.87
}

\subsubsect{(ii) $1.5v, 3\ GHz$}

\eq{
P &= \frac{1}{2} (30) (1.5)^2 \times 3 \\
P &= 101.25
}

\subsect{(16) Fabrication Yield}

\eq{
Yield = \frac{Working\ Chips}{Chips\ on\ Wafer}
}

Yield decreases as Chip size decreases relative to the wafer size.

\subsect{(17) Fabrication Cost 2}

\eq{
Chip\ Cost = \frac{Wafer\ Cost}{\#\ of\ Good\ Chips/Wafer}
}

\sect{Lesson 3: Metrics and Evaluation}

\subsect{(2) Performance}

\bullets{
\item
Latency (start $\rightarrow$ Done)
\item
Throughput (\# / second)
}

\eq{
	Throughput = \frac{1}{Latency}
}

\subsect{(3) Quiz: Latency and Throughput}

\bullets{
\item
2 servers
\item
A server takes 1 msec to process an order
}

\eq{
Throughput &= 2000\ orders/sec \\
Latency &= 1\ msec
}

\subsect{(4) Comparing Performance}

\eq{
Speedup &= N \\
N &= \frac{Speed(x)}{Speed(y)} = \frac{Latency(y)}{Latency(x)} = \frac{Throughput(x)}{Throughput(y)}
}

\subsect{(5) Quiz: Performance Comparison 1}

\bullets{
\item
Laptop takes 4 hours to compress video
\item
New laptop takes 10 minutes
}

\eq{
Speedup = \frac{(4 \times 60)}{10} = 24
}

\subsect{(6) Quiz: Performance Comparison 2}

The new laptop broke, so you have to go back to using the slower laptop.

\eq{
Speedup = \frac{10}{(4 \times 60)} = 0.042
}

\subsect{(7) Speedup}

\eq{
Speedup > 1 &\Rightarrow Improved\ Performance \\
Speedup < 1 &\Rightarrow Worse\ Performance \\
Performance &\sim Throughput \\
Performance &\sim \frac{1}{Latency}
}

\subsect{(13) Summarizing Performance}

Use the \emph{geometric mean} instead of the \emph{arithmetic mean} for comparing speedups.

\img{figures/lesson3_13.png}

\subsect{(14) Quiz: Speedup Averaging}

\bullets{
\item
Speedup of 2
\item
Speedup of 8
}

\eq{
Overall\ Speedup = \sqrt{2 \times 8} = 4
}

\subsect{(15) Iron Law of Performance}

\eq{
CPU\ Time &= \#\ of\ Instructions \\
&\times Cycles\ per\ Instruction \times Clock\ Cycle\ Time
}

\subsect{(16) Quiz: Iron Law 1}

\bullets{
\item
3 Billion Instructions
\item
2 Cycles / Instruction
\item
Processor Clock is 3 GHz
}

\eq{
	Execution\ Time = 3\ billion \times 2 \times \frac{1}{3\ billion} = 2\ sec
}

\subsect{(17) Iron Law for Unequal Instruction Times}

\eq{
CPU\ Time = \left(\sum_{i} = IC_i \times CPI_i\right) \times frequency
}

\subsect{Quiz: Iron Law 2}

\bullets{
\item
10 billion branches (CPI = 4)
\item
15 billion loads (CPI = 2)
5 billion stores (CPI = 3)
\item
20 billion integer adds (CPI = 1)
\item
Clock at 4 GHz
}

\eq{
Execution\ Time = \frac{10b \times 4 + 15b \times 2 + 5b \times 3 + 20b \times 1}{4b} = 26.25\ sec
}

\subsect{(19) Amdahl's Law}

Calculates the overall speedup when a portion of a program is enhanced.

\eq{
Frac_{ENH} &= \%\ of\ original\ execution\ time\ that\ is\ af\!fected\ by\ the\ enhancement \\
Speedup &= \frac{1}{(1 - Frac_{ENH}) + \frac{Frac_{ENH}}{Speedup_{ENH}}}
}

\subsect{(20) Quiz: Amdahl's Law}

\bullets{
\item
50 billion instructions
\item
2 GHz
\item
Improve branch from 4 to 2 CPI
}

\img{figures/lesson3_20.png}

\eq{
Frac_{ENH} &= \frac{0.2 \times 4}{0.4 \times 1 + 0.2 \times 4 + 0.3 \times 2 + 0.1 \times 3} \approx 0.381 \\
Speedup &= \frac{1}{(1 - 0.381) + \frac{0.381}{\frac{4}{2}}} \approx 1.24
}

\pagebreak

\subsect{(22) Quiz: Amdahl's Law 2}

\img{figures/lesson3_22.png}

\subsubsect{(i) Branch CPI 4 $\rightarrow$ 3}

\eq{
Speedup = \frac{1}{0.8 + \frac{0.2}{4/3}} = 1.05
}

\subsubsect{(ii) Increase Clock Frequency 2 $\rightarrow$ 2.3 GHz}

\eq{
Speedup = \frac{1}{0 + \frac{1}{2.3/2}} = 1.15
}

\subsubsect{(iii) Store CPI 3 $\rightarrow$ 2}

\eq{
Speedup = \frac{1}{0.9 + \frac{0.1}{3/2}} = 1.034
}

\pagebreak

\subsect{Lhamda's Law}

\bullets{
\item
Amdahl: Make common case fast
\item
Lhamda: Do not mess up uncommon case too badly
}

Example

\eq{
Speedup = \frac{1}{\frac{0.1}{0.1} + \frac{0.9}{2}} = 0.7
}

\subsect{(27 - 37) Problem Set}

\sect{Lesson 4: Pipelining}

\subsect{(3) Pipelining in a Processor}

\img{figures/lesson4_3.png}

\subsect{(4) Quiz: Laundry Pipelining}

\bullets{
\item
Wash (1 hour) $\rightarrow$ Dry (1 hour) $\rightarrow$ Fold (1 hour)
\item
10 loads of laundry
}

\eq{
No\ Pipelining &\Rightarrow 3 \times 10\ hours = 30\ hours \\
With\ Pipelining &\Rightarrow 1 \times 3\ hours + 9 \times 1\ hour = 12\ hours
}

\subsect{(5) Instruction Pipelining}

\bullets{
\item
5-stage pipeline (1 cycle/stage)
\item
10 instructions
}

\eq{
No\ Pipelining \Rightarrow 10 \times 5\ cycles = 50\ cycles \\
With\ Pipelining \Rightarrow 1 \times 5\ cycles + 9 \times 1\ cycle = 14\ cycles
}

\pagebreak

\subsect{(8) Processor Pipeline Stalls and Flushes}

\img{figures/lesson4_8.png}

\subsect{(10) Quiz: Control Dependencies}

\bullets{
\item
25\% of all instructions are taken branches with jumps
\item
10-stage pipeline
\item
Correct target for branch/jump computed in $6^{th}$ stage
}

\eq{
Actual\ CPI = 0.75 + 0.25 \times 6 = 2.25
}

subsect{(11) Data Dependencies}

\bullets{
\item
Read after write $\rightarrow$ True Dependency
\item
Write after write and write after read $\rightarrow$ False Dependency
}

\subsect{(12) Quiz: Data Dependencies}

I1: MUL R1,R2,R3 \\
I2: ADD R4,R4,R1 \\
I3: MUL R1,R5,R6 \\
I4: SUB R4,R4,R1

\tabl{c | c | c | c}{
& RAW & WAR & WAW \\ \hline
I1 $\rightarrow$ I2 & $\checkmark$ & - & - \\ \hline
I1 $\rightarrow$ I3 & - & - & $\checkmark$ \\ \hline
I1 $\rightarrow$ I4 & - & - & - \\ \hline
I2 $\rightarrow$ I3 & - & $\checkmark$ & - \\
}

\subsect{(13) Dependencies and Hazards}

\bullets{
\item
Dependencies - Instructions in a program that share the same registers.
\item
Hazards - Dependencies that can result in incorrect execution.
}

\subsect{(14) Quiz: Dependencies and Hazards}

\tabl{l | c | c}{
ADD R1,R2,R3 & Dependency? & Hazard? \\ \hline
SUB R5,R1,R4 & $\checkmark$ & $\checkmark$ \\ \hline
DIV R6,R1,R7 & $\checkmark$ & - \\ \hline
MUL R7,R1,R8 & $\checkmark$ & - \\
}

\subsect{(15) Handling Hazards}

\bullets{
\item
Detect hazardous situations
\bullets{
\item
Flush dependent instructions
\item
Stall dependent instructions
\item
Fix values read by dependent instructions
}
}

\subsect{(16) Quiz: Flushes, Stalls, and Forwarding}

\eq{
Fetch \rightarrow Read \rightarrow ALU/BR \rightarrow Mem \rightarrow Wr
}

\tabl{c  l | c | c | c}{
& BNE R1,R0,Label & Flush & Stall & Forward \\ \hline
& ADD R4,R5,R6 & \multirow{2}{*}{$\checkmark$} & \multirow{2}{*}{-} & \multirow{2}{*}{-} \\
& SUB R5,R4,R3 & & & \\ \hline
& MUL R1,R2,R3 & \multirow{2}{*}{-} & \multirow{2}{*}{-} & \multirow{2}{*}{$\checkmark$} \\
Label: & LW R1,0(R1) & & & \\ \hline
& ADD R1,R1,R1 & - & $\checkmark$ & $\checkmark$ \\
}

\pagebreak

\subsect{(17) How Many Stages?}

\bullets{
\item
Ideal CPI = 9
\item
More stages means:
\bullets{
\item
$More\ Hazards \Rightarrow CPI \uparrow$
\item
$Less\ Work/Stage \Rightarrow Cycle Time \downarrow$
}
}

\eq{
Exe\ Time &= \#Inst \times Cycle\ Time \\
\#\ Stages &\Rightarrow Balance\ CPI\ \&\ Cycle\ Time \\
Performance &\Rightarrow 30 - 40\ stages \\
Performance + Power &\Rightarrow 10 - 15\ stages
}

\img{figures/lesson4_17.png}

\subsect{(20 - 30) Problem Set}

\sect{Lesson 5: Branches}

\subsect{(3) Branch Prediction Requirements}

\bullets{
\item
Know only PC of instruction
\bullets{
\item
What is the PC of the next instruction to fetch
}
\item
Must Correctly Guess:
\bullets{
\item
Is this a taken branch?
\item
If taken, what is the target PC?
}
}

\subsect{(4) Branch Prediction Accuracy}

\eq{
CPI = 1 + \frac{mispreds}{inst} \times \frac{penalty}{mispred}
}

\bullets{
\item
20\% of all instructions are branches
}

\tabl{c | c | c}{
Accuracy $\downarrow$ & Resolve BR in 3$^{rd}$ Stage & Resolve BR in 10$^{th}$ Stage \\ \hline
50\% for BR & $1 + 0.5 \times 0. \times 2$ & $1 + 0.5 + 0.2 \times 9$ \\
100\% for other & 1.2 & 1.9 \\ \hline
90\% for BR & $1 + 0.1 \times 0.2 \times 2$ & $1 + 0.1 \times 0.2 \times 9$ \\
100\% for all other & 1.04 & 1.18 \\ \hline
Speedup & 1.15 & 1.61 \\
}

\pagebreak

\subsect{(5) Quiz: Branch Prediction Benefit}

\bullets{
\item
5 stage pipeline
\item
Branch resolved in 3$^{rd}$ stage
\item
Fetch nothing until sure what to frtch
\item
Execute many iterations of:
}

\tabl{l | c | c}{
LOOP: & Cost No Pred & Perfect Pred \\ \hline
ADD R1,R1,-1 & 2 & 1 \\ \hline
ADD R2,R2,R2 & 2 & 1 \\ \hline
BNEZ R1,LOOP & 3 & 1 \\ \hline
Total & 7 & 3 \\
}

\eq{
Speedup = \frac{7}{3} \approx 2.33
}

\subsect{(6) Performance with Not-taken Prediction}

\bullets{
\item
Refuse to predict
\bullets{
\item
Branch: 3 cycles
\item
Non-branch: 2 cycles
}
\item
Predict not-taken:
\bullets{
\item
Branch: 1 or 3 cycles
\item
Non-branch: 1 cycle
}
}

\pagebreak

\subsect{Quiz: Multiple Predictions}

\eq{
	Fetch \rightarrow Decode \rightarrow ALU\ (BR\ Resolved) \rightarrow Mem \rightarrow WR
}

\bullets{
\item
Using not-taken predictor
}

\noindent
BNE R1,R2,LABEL A \quad (Taken) \\
BNE R1,R3,LABEL B \quad (Taken) \\
A \\
B \\
C \\
LABEL A: \\
X \\
Y \\
LABEL B: \\
Z

\eq{
Cycles\ Wasted\ on\ Mispredictions = 2
}

\subsect{(8) Predict Not-Taken}

\bullets{
\item
Increment PPC
\item
20\% of instructions are branches
\item
60\% of branches are taken
\item
Correct: 80\% + 8\%, \quad Incorrect: 12\%
}

\eq{
CPI = 1 + 0.12 \times Penalty
}

\subsect{(9) Why We Need Better Prediction}

\tabl{c | c | c | c}{
& Not Taken (88\%) & Better (99\%) & Speedup \\ \hline
5-stage pipe & $1 + 0.12 \times 2$ & $1 + 0.01 \times 2$ & \multirow{2}{*}{1.22} \\
(3$^{rd}$ stage) & 1.24 & 1.02 & \\ \hline
14-stage & $1 + 0.12 \times 10$ & $1 + 0.01 \times 10$ & \multirow{2}{*}{2} \\
(11$^{th}$ stage) & 2.2 & 1.1 & \\ \hline
11$^{th}$ stage & $0.25 + 0.12 \times 10$ & $0.25 + 0.01 \times 10$ & \multirow{2}{*}{4.14} \\
4 inst/cycle & 1.45 & 0.35 & \\
}

\subsect{(10) Quiz: Predictor Impact}

\bullets{
\item
Pentium 4 "Prescott":
\bullets{
\item
FETCH, ... 29 stages ... , Resolve branches
\item
Branch prediction
\item
Multiple instruction/cycle
}
\item
Program:
\bullets{
\item
20\% of instructions are branches
\item
1\% if branches are mispredicted
\item
CPI = 0.5
}
}

Perfect branch predictor:
\eq{
CPI = 0.5 - 0.2 \times 0.01 \times 30 = 0.44
}

If 2\% of branches are mispredicted:
\eq{
CPI = 0.44 + 0.2 \times 0.02 \times 30 = 0.56
}

\subsect{(12) Better Prediction -- How?}

\eq{
PC_{next} = f(PC_{now}, History[PC_{now}])
}

\subsect{(13) BTB -- Branch Target Buffer}

\img{figures/lesson5_13.png}

\subsect{(14) Realistic BTB}

\bullets{
\item
Map least significant bits in $PC_{now}$ to an index in the BTB
}

\subsect{(15) Quiz: BTB}

\bullets{
\item
BTB has 1024 entries
\item
Fixed-size 4-byte instructions, word-aligned
\item
32-bit address
\item
$PC = 0x0000AB0C$
}

\eq{
BTB\ Index = 10leastSigUniqBits(PC) = 0b1011000011 = 0x2C3
}

\subsect{(16) Direction Predictor}

\img{figures/lesson5_16.png}

\pagebreak

\subsect{(17 - 21) Quiz: BTB and BHT}

\bullets{
\item
BHT: 16 entries, perfect prediction
\item
BTB: 4 entries, perfect prediction
}

\scriptsize
\tabl{l r l | c | c | c | c | c}{
& & & & & & & Missed \\
& & & Times & & Times & BTB & Preds. \\
& & & BHT & BHT & BTB & Entry & if 1-bit \\
0xC000 & & MOV R2,100 & Accessed & Entry & Accessed & Used & BHT \\ \hline
0xC004 & & MOV R1,0 & 1 & 0 & 0 & & \\ \hline
0xC008 & LOOP: & BEQ R1,R2,DONE & 101 & 2 & 1 & 2 & 1 \\ \hline
0xC00C & & ADD R4,R3,R1 & 100 & 3 & 0 & & \\ \hline
0xC010 & & LW R4,0(R4) & 100 & 4 & 0 & & \\ \hline
0xC014 & & ADD R5,R5,R4 & 100 & 5 & 0 & & \\ \hline
0xC018 & & ADDR1,R1,1 & 100 & 6 & 0 & & \\ \hline
0xC01C & & MOV R1,0 & 100 & 7 & 100 & 3 & 1 \\ \hline
0xC020 & DONE: & & & & & & \\
}
\normalsize

\subsect{(22) Problems with 1-Bit Prediction}

\bullets{
\item
Predicts well:
\bullets{
\item
Always taken
\item
Always not taken
\item
$Taken >>> Not\ Taken$
\item
$Taken <<< Not\ Taken$
}
\item
Not so well:
\bullets{
\item
$Taken > Not\ Taken$
\item
$Not\ Taken > Taken$
\item
Short loop
}
\item
Pretty bad:
\bullets{
\item
$Taken \approx Not\ Taken$
}
}

\eq{
Each\ Anomoly \Rightarrow Two\ Mispredictions
}

\subsect{(23) 2-Bit Predictor}

\bullets{
\item
00: Strong Not Taken
\item
01: Weak Not Taken
\item
10: Weak Taken
\item
11: Strong Taken
}

\img{figures/lesson5_23.png}

\subsect{(25) Quiz: 2BP}

\bullets{
\item
2-bit predictor
\item
Start in strong not-taken state
\item
A sequence always resulting in incorrect prediction:
\center{
T T N T N ...
}
}

\subsect{(26) 1BP,2BP}

\bullets{
\item
$1BP \rightarrow 2BP$ \quad 3BP? \quad 4BP?
\bullets{
\item
Bad: Cost $\uparrow$
\item
Good: When "anomalous" outcomes come in streaks
\bullets{
\item
How often does this happen?
}
}
\item
Stay with 2BP, maybe 3BP
}

\subsect{(28) 1-Bit History with 2-Bit Counters}

\tabl{c | c | c | c}{
State & Pred & Outcome & Correct? \\ \hline
(0, SN, SN) & N & T & $\times$ \\ \hline
(1, WN, SN) & N & N & $\checkmark$ \\ \hline
(0, WN, SN) & N & T & $\times$ \\ \hline
(1, WT, SN) & N & N & $\checkmark$ \\ \hline
(0, WT, SN) & T & T & $\checkmark$ \\ \hline
(1, ST, SN) & N & N & $\checkmark$ \\ \hline
(0, ST, SN) & T & T & $\checkmark$ \\
}

\subsect{(29) Quiz: 1-Bit History}

\bullets{
\item
1-Bit history (Start 0)
\item
2-BC/history (Start SN)
\item
(NNT)*
\item
Continues for 100 repitions
}

\eq{
\#\ Mispredictions = 100
}

\subsect{(30 - 31) N-Bit History Predictors}

\bullets{
\item
Works for all patterns of $length \leq N + 1$
\item
Costs $N + 2 \times 2^{N}$ per entry
\item
Most BCs are wasted
}

\pagebreak

\subsect{(32) Quiz: N-Bit History Predictor}

\bullets{
\item
N-Bit history, 2BCs/History
\item
Need 1024 entries
}

\tabl{c | c | c | c}{
& & & \# of 2BCs \\
& & & used for \\
& cost(bits) & (NNNT)* & (NT)* \\ \hline
N = 1 & $5 \times 1024$ & bad & 2 \\ \hline
N = 4 & $(4 + 2 \times 2^4) \times 1024$ & good & 2 \\ \hline
N = 8 & 532,480 & good & 2 \\ \hline
N = 16 & 134,234,112 & good & 2 \\
}

\subsect{(33) Quiz: History Predictor}

for(int i = 0; i != 8; i++) \\
\indent for(int j = 0; j != 8; j++) \\
\indent \indent doSomething();

\bullets{
\item
The history predictor needs at least 4 entries
\item
Each entry needs at least an 8-bit history (246 2-bit counters)
}

\subsect{(34 - 36) History with Shared Counters}

\img{figures/lesson5_34.png}

\eq{
TTTT &\Rightarrow 1111 \Rightarrow 1\ counter \\
NNNN &\Rightarrow 0000 \Rightarrow 1\ counter \\
NTNT &\Rightarrow 010101\ or\ 101010 \Rightarrow 2\ counters \\
NNNNT &\Rightarrow 16 \times H \Rightarrow 16\ counters
}

\subsect{(37) PShare}

PShare

\bullets{
\item
Private history
\item
Shared counters
\item
Good for:
\bullets{
\item
Even-odd
\item
8-iteration loop
}
}

\noindent
GShare

\bullets{
\item
Global history
\item
Shared counters
\item
Good for:
\bullets{
\item
Correlated branches
}
}

\subsect{(38) Quiz: PShare vs. GShare}

for(int i = 1000; i != 0; i--) \\
\indent if(i \% 2) \\
\indent \indent n += i; \\
\indent \indent \indent $\Downarrow$

\noindent
LOOP: \\
BEQ R1,0,EXIT \\
AND R2,R1,1 \\
BEQ R2,0,EVEN \\
ADD R3,R3,R1 \\
ADD R1,R1,-1 \\
B LOOP \\
EXIT:

\eq{
History\ for\ PShare &= 1 \\
History\ for\ GShare &= 3
}

\subsect{(40) Tournament Predictor}

\bullets{
\item
Two Predictors
\bullets{
\item
One better for branches x, y, z
\item
Other better for branches a, b, c
}
}

\img{figures/lesson5_40.png}

\subsect{(41) Hierarchical Predictor}

\bullets{
\item
Tournament
\bullets{
\item
2 good predictors
\item
update both on each dicision
}
\item
Hierarchical
\bullets{
\item
1 good, 1 ok predictor
\item
Update ok-predictor on each decision
\item
Update good-predictor onlu if the ok-predictor is not sufficient.
}
}

\subsect{(43) Quiz: Multi-Predictor}

\bullets{
\item
2 BP works fine for 95\% of instructions
\item
PShare works fine for the same 95\%, plus 2\% more
\item
GShare works fine for the same 95\%, plus the other 3\%
}

\noindent
The overall predictor is a Hierarchical predictor that chooses between a 2BP and a tournament predictor, which itself chooses between a PShare and a GShare.

\subsect{(45) RAS}

\bullets{
\item
Contains the return addresses of functions
\item
When the RAS is full, there are two choices:
\bullets{
\item
Don't push
\item
Wrap around
}
}

\subsect{(46) Quiz: RAS Full}

When the RAS is full, it is better to wrap around than to not push.

\subsect{(47) But How Do We Know it's a RET}

\bullets{
\item
Predictor
\bullets{
\item
Must be very accurate
\item
PC = 0xABCD was RET $\Rightarrow$ 0xABCD is RET now
}
\item
Precode
\bullets{
\item
Annotate instruction in cache
}
}

\subsect{(49 - 53) Problem Set}

\sect{Lesson 6: Predication}

\subsect{(2) Prediction}

Branch Prediction

\bullets{
\item
Guess where it is going
\item
No penalty if correct (0\% waste)
\item
Huge panalty if wrong (50\% waste)
}

Predication

\bullets{
\item
Do work of both directions
\item
Wast up to 50\%
\item
Throw away work from wrong path
}

\noindent
Loop $\Rightarrow$ Predict \\
Func $\Rightarrow$ Predict \\
Large if-then-else $\Rightarrow$ Predict \\
Small if-then-else $\Rightarrow$ Predicate

\subsect{(3) If Conversion}

if(cond)\{ \\
\indent x = arr[i]; \\
\indent y = y + 1; \\
\}else\{ \\
\indent x = arr[j]; \\
\indent y = y - 1; \\
\}

\indent \indent $\Downarrow$

\noindent
x1 = arr[i]; \\
x2 = arr[j]; \\
y1 = y + 1; \\
y2 = y - 1; \\
x = cond ? x1 : x2; $\Rightarrow$ MOV x,x1,cond \\
y = cond ? y1 : y2; $\Rightarrow$ MOV y,y1,cond

\subsect{(4) Conditional Move}

MIPS
\bullets{
\item
MOVZ $R_d,R_s,R_t \Rightarrow$ if($R_t$ == 0) $R_d = R_s$;
\item
MOVN $R_d,R_s,R_t \Rightarrow$ if($R_t$ != 0) $R_d = R_s$;
}

\noindent
x86
\bullets{
\item
CMOVZ, CMOVNZ, CMOVGT, ... $\Rightarrow$ if(cond) Dst = Src;
}

\subsect{(5, 7) Quiz: MOVZ, MOVN, and Performance}

BEQZ R1,ELSE (taken 50\% of the time) \\
ADDI R2,R2,1 \\
B END \\
ELSE: \\
ADDI R3,R3,1 \\
END: \\
\indent \indent $\Downarrow$ \\
ADDI R4,R2,1 \\
ADDI R5,R3,1 \\
MOVN R2,R4,R1 \\
MOVZ R3,R5,R1

\bullets{
\item
30-instruction penalty
}

\eq{
Performance\ Threshold = \frac{4 - (2 + 3) \times 0.5}{30} = 5\%\ Accuracy
}

\noindent
If-conversion is better when prediction accuracy is below 95\%.

\pagebreak

\subsect{(8) MOVc Summary}

\bullets{
\item
Needs compiler support
\item
Removes hard-to-predict branches
\item
More registers are needed (unless full predication is possible)
\bullets{
\item
Results from both paths
}
\item
More instructions are executed
\bullets{
\item
Both paths
\item
MOVc to select actual results (unless full predication is possible)
}
}

\noindent
Full predication is achieved when all instructions are conditional. This requires extensive support in the instruction set.

\subsect{(9) Full Predication HW Support}

MOVcc
\bullets{
\item
Separate opcode
}

\noindent
Full Predication
\bullets{
\item
Add condition bits to every instruction
}

\subsect{(10) Full Predication Example}

BEQZ R1,ELSE \\
ADDI R2,R2,1 \\
B END \\
ELSE: \\
ADDI R3,R3,1 \\
END: \\
\indent \indent $\Downarrow$ (if-convert)\\
MP.EQZ P1,P2,R1 \\
(P2) ADDI R2,R2,1 \\
(P1) ADDI R3,R3,1

\subsect{(11) Quiz: Full Predication}

BEQZ R2,ELSE \\
ADD R1,R1,1 \\
B DONE \\
ELSE: \\
ADD R1,R1,-1 \\
DONE: \\
\indent \indent $\Downarrow$ (if-convert) \\
MP.EQZ P1,P2,R2 \\
(P2) ADD R1,R1,1 \\
(P1) ADD R2,R2,1

\bullets{
\item
CPI = 0.5 without mispredictions
\item
Misprediction penalty of 10 cycles
}

\eq{
Performance\ Threshold &= \frac{(3 \times 0.5) - (2 + 3) \times 0.5 \times 0.5}{10} \\
&= 2.5\%\ Accuracy
}

\noindent
Do if-conversion if BEQZ prediction is less than 97.5\% correct.

\subsect{(13 - 32) Problem Set}

\sect{Lesson 7: ILP}

\subsect{(2) ILP All in the Same Cycle}

\tabl{l | c | c | c | c | c}{
Instruction & Cycle 1 & 2 & 3 & ... & N \\ \hline
R1 = R3 + R3 & FETCH & DECODE & EXECUTE & ... & WRITE \\ \hline
R4 = R1 - R5 & FETCH & DECODE & EXECUTE & ... & WRITE \\ \hline
R6 = R7 $\oplus$ R8 & FETCH & DECODE & EXECUTE & ... & WRITE \\ \hline
R5 = R8 $\times$ R9 & FETCH & DECODE & EXECUTE & ... & WRITE \\ \hline
R4 = R8 + R9 & FETCH & DECODE & EXECUTE & ... & WRITE \\
}

\eq{
CPI = \frac{N\ cycles}{\infty\ instructions} = 0
}

\bullets{
\item
Data dependencies make this impossible
}

\subsect{(3) The Execute Stage}

\bullets{
\item
I2 depends on I1
}

\tabl{c | c}{
& EXE \\ \hline
I1 & I1 \\ \hline
I2 & stall \\ \hline
I3 & I3 \\ \hline
I4 & I4 \\ \hline
I5 & I5 \\
}

\eq{
CPI = \frac{2\ cycles}{5\ instructions} = 0.4
}

\pagebreak

\subsect{(6) Quiz: Dependency}

\bullets{
\item
5-stages (FETCH, RR, EXE, MEM, WREG)
\item
Forwarding enabled
\item
10 inputs in each stage
}

\tabl{l | c | c}{
& EXE Cycle & WB \\ \hline
MUL R2,R2,R2 & 2 & 4 \\ \hline
ADD R1,R1,R2 & 3 & 5 \\ \hline
MUL R3,R3,R3 & 2 & 4 \\ \hline
ADD R1,R1,R3 & 4 & 6 \\ \hline
MUL R4,R4,R4 & 2 & 4 \\ \hline
ADD R1,R1,R4 & 5 & 7 \\
}

\subsect{(7) Removing False Dependencies}

\bullets{
\item
RAW -- True Dependencies
\item
WAR, WAW -- False (Name) Dependencies
}

\subsect{(8) Duplicating Register Values}

\tabl{l | c | c | c | c}{
& c100 & c101 & c102 & c103 \\ \hline
R1 = R2 + R3 & EXE & MEM & WR & - \\ \hline
R4 = R1 - R5 & - & EXE & MEM & WR \\ \hline
R3 = R4 + 1 & - & - & EXE & MEM \\ \hline
R4 = R8 - R9 & EXE & MEM & WR & - \\ \hline
\multicolumn{1}{| c |}{$\vdots$} & $\vdots$ & $\vdots$ & $\vdots$ & $\vdots$ \\ \hline
$\cdots$ = R4 $\cdots$ & $\cdots$ & $\cdots$ & $\cdots$ & $\cdots$ \\
}

\bullets{
\item
Create copies of shared registers
}

\pagebreak

\subsect{(9) Register Renaming}

\bullets{
\item
\emph{Architectural Registers}: Registers a programmer or compiler use
\item
\emph{Physical Registers}: All places a value can go
\item
Rewrite a program to use physical registers
\item
\emph{Register Allocation Table (RAT)}: A table that says which physical register has value for which architectural register
}

\subsect{(10) RAT Example}

\center

Initial RAT

\tabl{l | l}{
0 & P0 \\ \hline
1 & P1 \\ \hline
2 & P2 \\ \hline
3 & P3 \\ \hline
4 & P4 \\ \hline
5 & P5 \\ \hline
6 & P6 \\ \hline
$\vdots$ & $\cdots$ \\
}

Instructions

\tabl{l | l}{
Fetched & Renamed \\ \hline
ADD R1,R2,R3 & ADD P17,P2,P3 \\ \hline
SUB R4,R1,R5 & SUB P18,P17,P5 \\ \hline
XOR R6,R7,R8 & XOR P19,P7,P8 \\ \hline
MUL R5,R8,R9 & MUL P20,P8,P9 \\ \hline
ADD R4,R8,R9 & ADD P21,P8,P9 \\
}

\pagebreak

\flushleft

\subsect{(11) Quiz: Register Renaming}

\center

Final RAT

\tabl{l | l}{
R1 & P8 \\ \hline
R2 & P11 \\ \hline
R3 & P10 \\ \hline
R4 & P4 \\ \hline
R5 & P12 \\ \hline
R6 & P6 \\
}

Instructions

\tabl{l | l}{
Fetched & Renamed \\ \hline
MUL R2,R2,R2 & MUL P7,P2,P2 \\ \hline
ADD R1,R1,R2 & ADD P8,P1,P7 \\ \hline
MUL R2,R4,R4 & MUL P9,P4,P4 \\ \hline
ADD R3,R3,R2 & ADD P10,P3,P9 \\ \hline
MUL R2,R6,R6 & MUL P11,P6,P6 \\ \hline
ADD R5,R5,R2 & ADD P12,P5,P11 \\
}

\flushleft

\subsect{(13) Instruction Level Parallelism (ILP)}

ILP = IPC When:

\bullets{
\item
Processor does entire instruction in 1 cycle
\item
Processor can do any number of instructions in the same cycle
\item
Has to obey true dependencies
\item
ILP is a property of a program, not a processor.
}

Steps to Get ILP:
\enum{
\item
Rename registors
\item
"Execute"
}

\subsect{(14) ILP Example}

\tabl{l | c}{
Instruction & Cycle \\ \hline
ADD P10,P2,P3 & 1 \\ \hline
XOR P6,P7,P8 & 1 \\ \hline
MUL P5,P8,P9 & 1 \\ \hline
ADD P4,P8,P9 & 1 \\ \hline
SUB P11,P10,P5 & 2 \\
}

\eq{
ILP = \frac{5\ instructions}{2\ cycles} = 2.5
}

\subsect{(15) Quiz: ILP}

\tabl{l | c}{
Instruction & Cylcle \\ \hline
ADD R1,R1,R1 & 1  \\ \hline
ADD R2,R2,R1 & 2 \\ \hline
ADD R3,R2,R1 & 3 \\ \hline
ADD R6,R7,R8 & 1 \\ \hline
ADD R8,R3,R7 & 4 \\ \hline
ADD R1,R1,R1 & 2 \\ \hline
ADD R1,R7,R7 & 1 \\
}

\eq{
ILP = \frac{7\ instructions}{4\ cycles} = 1.75
}

\subsect{(16) ILP With Structural \& Control Dependencies}

\bullets{
\item
No structural dependencies
\item
Perfect same-cycle branch prediction
}

\subsect{(17) ILP vs. IPC}

\bullets{
\item
ILP does not consider processor limitatipns, while IPC does
\item
ILP $\geq$ IPC
}

\subsect{(18) Quiz: IPC \& ILP}

\bullets{
\item
3-issue in-order
\item
3 ALUs
}

\tabl{l | c | c}{
Instruction & ILP Cylce & IPC Cycle \\ \hline
ADD R1,R2,R3 & 1 & 1 \\ \hline
ADD R2,R3,R4 & 1 & 1 \\ \hline
ADD R3,R1,R2 & 2 & 2 \\ \hline
ADD R7,R8,R9 & 1 & 2 \\ \hline
ADD R1,R7,R7 & 2 & 2 \\ \hline
ADD R1,R4,R5 & 1 & 3 \\
}

\eq{
ILP &= \frac{6\ instructions}{2\ cycles} = 3 \\
IPC &= \frac{6\ instructions}{3\ cycles} = 2
}

\subsect{(21 - 24) Problem Set}

\sect{Lesson 8: Instruction Scheduling}

\subsect{(2) Improving IPC}

\bullets{
\item
ILP can be good ($>>$ 1)
\item
Control dependencies $\Rightarrow$ branch prediction
\item
WAR \& WAW data dependencies $\Rightarrow$ register renaming
\item
RAW data dependencies $\Rightarrow$ out of order execution
\item
Structural dependencies $\Rightarrow$ invest in wider-issue
}

\subsect{(3) Tomasulo's Algorithm}

\bullets{
\item
Used in IBM 360
\item
Determine which instructions have inputs ready
\item
Register renaming
\item
Very similar to what we use today
}

Tomasulo:

\bullets{
\item
Floating point instructions
\item
Fewer instructions in "window"
\item
No exception handling
}

Today:

\bullets{
\item
All instructions
\item
100s of instructions
\item
Exception handling support
}

\pagebreak

\subsect{(4) The Picture}

\img{figures/lesson8_4.png}

\subsect{(5) Issue}

\bullets{
\item
Take next (in program order) instruction from Issue Queue (IQ)
\item
Determine where inputs come from
\item
Get free reservation station (or correct kind)
\item
Put instruction into reservation station
\item
Tag destination register
}

\pagebreak

\subsect{(6) Issue Examples}

\center

\cols{
IQ

\tabl{c | l}{
4 & F1 = F2 + F3 \\ \hline
3 & F4 = F1 - F2 \\ \hline
2 & F1 = F2 / F3 \\ \hline
1 & F2 = F4 + F1 \\
}
}{
RAT

\tabl{c | c}{
F1 & RS3 \\ \hline
F2 & FS1 \\ \hline
F3 & \\ \hline
F4 & RS2 \\
}
}

\cols{
ADD

\tabl{c | c | c | c}{
RS1 & + & 0.71 & 3.14 \\ \hline
RS2 & - & RS4 & RS1 \\ \hline
RS3 & + & RS1 & 2.72 \\
}
}{
MUL

\tabl{c | c | c | c}{
RS4 & / & RS1 & 2.72 \\ \hline
RS5 & & & \\
}
}

RF

\tabl{c | c}{
F1 & 3.14 \\ \hline
F2 & -1.00 \\ \hline
F3 & 2.72 \\ \hline
F4 & RS2 \\
}

\flushleft

\pagebreak

\subsect{(7) Quiz: Issue}

\center

\cols{
IQ

\tabl{c | l}{
2 & F4 = F3 $\times$ F4 \\ \hline
1 & F4 = F1 / F2 \\
}
}{
RAT

\tabl{c | c}{
F1 & RS4 \\ \hline
F2 & FS1 \\ \hline
F3 & \\ \hline
F4 & RS5 \\
}
}

\cols{
ADD

\tabl{c | c | c | c}{
RS1 & ADD & 0.71 & -1 \\ \hline
RS2 & & & \\ \hline
RS3 & & & \\
}
}{
MUL

\tabl{c | c | c | c}{
RS4 & DIV & RS1 & 2.72 \\ \hline
RS5 & DIV & RS4 & RS1 \\
}
}

RF

\tabl{c | c}{
F1 & 3.14 \\ \hline
F2 & -1.00 \\ \hline
F3 & 2.72 \\ \hline
F4 & 0.71 \\
}

\flushleft

\subsect{(8) Dispatch}

When a result is being broadcast:

\enum{
\item
Search for a matching tag
\item
Broadcast result
}

\subsect{(9) More than 1 Instruction Ready}

\bullets{
\item
Oldest first
\item
Most dependencies first
\item
Random
}

\subsect{(10) Quiz: Dispatch}

\bullets{
\item
Dispatching RS1 = 0.29
}

\center

ADD

\tabl{c | c | c | c}{
RS1 & & & \\ \hline
RS2 & SUB & RS1 & RS1 \\ \hline
RS3 & ADD & 1.23 & 2.72 \\
}

MUL

\tabl{c | c | c | c}{
RS4 & DIV & RS1 & 2.72 \\ \hline
RS5 & & & \\ \hline
}

\flushleft

RS3 wasn't dispatched already because it was either issued in the previous cycle or another instruction was dispatched in the ADD unit.

\subsect{(11) Write Result (Broadcast)}

\enum{
\item
Put Tag \& Res on bus
\item
Write to RF
\item
Update RAT
\item
Free RS
}

\subsect{(12) More than 1 Broadcast}

\bullets{
\item
Give priority to the slower operation
}

\subsect{(13) Broadcast Stale Results}

\bullets{
\item
Do not update the RAT or RF (Register File) with the result
}

\subsect{(19) Quiz: Tomasulo's Algorithm}

Tomasulo's Algorithm does not dispatch instructions or results in program order.

\subsect{(20) Load \& Store Instructions}

\bullets{
\item
Dependencies Through Memory
\bullets{
\item
RAW: SW to A, then LW from A
\item
WAR: LW then SW
\item
WAW: SW, then SW to same address
}
\item
What can we do?
\bullets{
\item
Do loads and stores in order
\item
Identify dependencies, reorder, etc.
}
}

\subsect{(27) Timing Example}

Load: 2 cycles \\
Add: 2 cycles \\
Multiply: 10 cycles \\
Divide: 40 cycles

\tabl{c | l | c | c | c | l}{
& \multicolumn{1}{ | c | }{Instruction} & IS & EX & WR & Comment \\ \hline
1 & L.D F6,34(R2) & 1 & 2 & 4 & \\ \hline
2 & L.D F2,45(R3) & 2 & 4 & 6 & \\ \hline
3 & MUL.D F0,F2,F4 & 3 & 7 & 17 & F2 from 2 \\ \hline
4 & SUB.D F8,F2,F6 & 4 & 7 & 9 & F2  from 2 \\ \hline
5 & DIV.D F10,F0,F6 & 5 & 18 & 58 & F0 from 3 \\ \hline
6 & ADD.D F6,F8,F2 & 6 & 10 & 12 & F8 from 4 \\
}

\pagebreak

\subsect{(28 - 29) Quiz: Tomasulo Timing}

Latency:

\bullets{
\item
LD: 1 cycle
\item
ADD: 1 cylce
\item
MUL: 5 cycles
}

\# of Reservation Stations:

\bullets{
\item
LD: 1
\item
ADD: 2
\item
MUL: 3
}

Same-cycle:

\bullets{
\item
Issue $\rightarrow$ Dispatch: No
\item
Capture $\rightarrow$ Dispatch: No
\item
RS freed $\rightarrow$ RS alloc: No
}

\tabl{c | l | c | c | c}{
1 & LD F6,0(R2) & 1 & 2 & 3 \\ \hline
2 & MUL F2,F0,F1 & 2 & 3 & 8 \\ \hline
3 & ADD F6,F2,F6 & 3 & 9 & 10 \\ \hline
4 & ADD F6,F2,F6 & 4 & 11 & 12 \\ \hline
5 & ADD F1,F1,F1 & 11 & 12 & 13 \\ \hline
6 & ADD F1,F3,F4 & 13 & 14 & 15 \\ \hline
}

\subsect{(31 - 35) Problem Set}

\sect{Lesson 9: ReOrder Buffer}

\subsect{(4) Correct Out of Order Execution}

\bullets{
\item
Execute 000
\item
Broadcast 000
\item
But deposit values to REGs in-order!
}

$\Rightarrow$ ReOrder Buffer:

\bullets{
	\item
	Remembers program order
	\item
	Keeps result until it is safe to write to REGs
}

\subsect{(5 - 6, 9) ROB}

\bullets{
\item
Issue: Get Registration Station, get ROB entry, point RAT to ROB entry
\item
Dispatch: Ready? Send to EXE and free RS!
\item
BCast: Capture, WR to ROB
\item
Commit: WR to REG, update RAT
}

\subsect{(7) Quiz: Free Reservation Station}

If there is no ROB, it is more likely for an instruction to be unable to issue because there is no available RS.

\subsect{(10) Quiz: ROB}

The ROB is needed to:

\bullets{
\item
Remember the program order
\item
Temporarily store an instruction's result
\item
Serve as the name (TAG) for the result
}

\subsect{(11) Branch Misprediction Recovery}

LD R1,0(R1) \\
BNE R1,R2,Label \\
ADD R2,R1,R1 \\
MUL R3,R3,R4 \\
DIV R2,R3,R7

\cols{
\center
REGS
\flushleft

\tabl{c | c}{
R1 & 700 \\ \hline
R2 & \\ \hline
R3 & \\
}
}{

\center
RAT
\flushleft

\tabl{c | c}{
\st{ROB1} & Issue \\ \hline
ROB5 & Wrong \\ \hline
ROB4 & Wrong \\
}
}

\tabl{l | c | c | c | c}{
& Line & REG & Val & Done \\ \hline
Commit [1] & 0 & & & \\ \hline
Commit [2] & 1 & R1 & 700 & $\checkmark$ \\ \hline
Commit [3], Issue [2] & 2 & - & ERROR & $\checkmark$ \\ \hline
& 3 & R2 & $\checkmark$ & $\checkmark$ (Wrong) \\ \hline
& 4 & R3 & 15 & $\checkmark$ (Wrong) \\ \hline
Issue [1] & 5 & R2 & 2 & $\checkmark$ (Wrong) \\
}

\subsect{(13) ROB and Exceptions}

\bullets{
\item
Treat an exception as a result
\item
Delay actual handling until the commit
}

\subsect{(15) Quiz:Exceptions with ROB}

\tabl{l | c | c}{
Instruction & Status & New Status \\ \hline
ADD R2,R2,R1 & committed & committed \\ \hline
LW R1,0(R2) & executing & committed \\ \hline
ADD R3,R4,R5 & done & committed \\ \hline
DIV R3,R2,R3 & executing (Exception) & unexecuted \\ \hline
ADD R1,R4,R4 & done & unexecuted \\ \hline
ADD R3,R2,R2 & done & unexecuted \\
}

\subsect{(31) ROB Timing Example}

\bullets{
\item
Free RS on BCast, not Dispatch
\item
Issue, Capture, Dispatch in same cycle
\item
ADD: 1 cycle, MUL: 10 cycles, DIV 40 cycles
}

\tabl{l | l | c | c | c | c | c}{
Inst & Operands & IS & EXE & WR & Commit & Comment \\ \hline
DIV & R2,R3,R4 & 1 & 2 & 42 & 43 & \\ \hline
MUL & R1,R5,R6 & 2 & 3 & 13 & 44 & \\ \hline
ADD & R3,R7,R8 & 3 & 4 & 5 & 45 & \\ \hline
MUL & R1,R1,R3 & 14 & 15 & 25 & 45 & Need RS $\rightarrow$ Issue \\ \hline
SUB & R4,R1,R5 & 15 & 25 & 27 & 47 & EXE depends on R1 \\ \hline
ADD & R1,R4,R2 & 16 & 43 & 44 & 48 & EXE depends on R2 \\
}

\subsect{(32 - 34) Quiz: ROB Timing}

\bullets{
\item
BCast one ADD/SUB and one MUL/DIV per cycle
\item
Commit up to 2 instructions per cycle
\item
Free RS at dispatch
\item
ADD: 1 cycle, MUL: 2 cycles, DIV: 4 cycles
}

\tabl{l | l | c | c | c | c}{
Inst & Operands & IS & EXE & WR & Commit \\ \hline
DIV & R2,R3,R4 & 1 & 2 & 6 & 7 \\ \hline
MUL & R1,R5,R6 & 2 & 3 & 5 & 7 \\ \hline
ADD & R3,R7,R8 & 3 & 4 & 5 & 8 \\ \hline
MUL & R1,R1,R2 & 4 & 7 & 9 & 10 \\ \hline
SUB & R4,R2,R5 & 5 & 7 & 8 & 10 \\ \hline
ADD & R1,R4,R3 & 6 & 9 & 10 & 11 \\
}

\subsect{(36) Superscalar}

\bullets{
\item
Fetch $>$ 1 inst/cycle
\item
Decode $>$ 1 inst/cycle
\item
Issue $>$ 1 inst/cycle
\item
Dispatch $>$ 1 inst/cycle
\item
Broadcast $>$ 1 result/cycle
\item
Commit $>$ 1 inst/cycle
}

\subsect{(37) Terminology Confusion}

\bullets{
\item
Issue $\Rightarrow$ Allocate, Dispatch
\item
Dispatch $\Rightarrow$ Execute, Issue
\item
Commit $\Rightarrow$ Complete, Retire, Graduate
}

\subsect{(38) Out of Order}

Only execution and broadcast stages occur out-of-order.

\subsect{(39) Quiz: In Order vs Out of Order}

\tabl{c | c | c}{
Stage & In-Order & Out-of-Order \\ \hline
Fetch & $\checkmark$ & \\ \hline
Decode & $\checkmark$ & \\ \hline
Issue & $\checkmark$ & \\ \hline
Dispatch & & $\checkmark$ \\ \hline
Execute 1 & & $\checkmark$ \\ \hline
Execute 2 & & $\checkmark$ \\ \hline
Broadcast & & $\checkmark$ \\ \hline
Commit & $\checkmark$ & \\ \hline
Release ROB Entry & $\checkmark$ & \\
}

\end{document}
